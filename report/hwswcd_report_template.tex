\documentclass{tudscrreprt}
% Documentation available at https://ctan.org/pkg/tudscr

\iftutex
\usepackage{fontspec}
\else
\usepackage[T1]{fontenc}
\fi

\usepackage[utf8]{inputenc}

\usepackage{hyperref}
\usepackage{cleveref}
\usepackage{listings}
\usepackage{color}
\usepackage{wrapfig}
\usepackage{tikz}
\usetikzlibrary[shapes,positioning]
\usepackage{chngcntr}
\counterwithout{figure}{chapter}
\usepackage{framed}

\lstset
{
    numbers=left,
    stepnumber=1,
    firstnumber=1,
    numberfirstline=true,
    basicstyle=\ttfamily,        % the size of the fonts that are used for the code
    breakatwhitespace=false,         % sets if automatic breaks should only happen at whitespace
    breaklines=true,                 % sets automatic line breaking
    captionpos=b,                    % sets the caption-position to bottom
    commentstyle=\color{cdgreen},    % comment style
    deletekeywords={...},            % if you want to delete keywords from the given language
    escapeinside={\%*(}{*)},          % if you want to add LaTeX within your code
    emphstyle=\bfseries,
    language=C,
    moredelim=[is][\bfseries]{\$\$\$}{\$\$\$},
    %extendedchars=true,              % lets you use non-ASCII characters; for 8-bits encodings only, does not work with UTF-8
    frame=single,	                   % adds a frame around the code
    keepspaces=true,                 % keeps spaces in text, useful for keeping indentation of code (possibly needs columns=flexible)
    keywordstyle=\color{cddarkblue},       % keyword style
    morekeywords={*,...},            % if you want to add more keywords to the set
    rulecolor=\color{black},         % if not set, the frame-color may be changed on line-breaks within not-black text (e.g. comments (green here))
    showspaces=false,                % show spaces everywhere adding particular underscores; it overrides 'showstringspaces'
    showstringspaces=false,          % underline spaces within strings only
    showtabs=false,                  % show tabs within strings adding particular underscores
    stringstyle=\color{cdpurple},     % string literal style
    tabsize=4,	                   % sets default tabsize to 2 spaces
    title=\lstname,                   % show the filename of files included with \lstinputlisting; also try caption instead of title
}

\lstdefinestyle{tie}{
    language=verilog,
    morekeywords={
        % Extracted from https://www.vim.org/scripts/script.php?script_id=1964
        state,regfile,immediate_range,
        table,
        interface,
        operation,
        schedule,semantic,length,
        format,slot,slot_opcodes,import_wire,
        queue,ctype,proto,user_register,coprocessor,
        function,field,opcode,operand,iclass,reference,
        add_read_write,assign,core,immediate,
        in,inout,Inst,InstBuf,MEM,out,wire,
        export,
        VAddr,
        MemDataIn128,MemDataIn64,MemDataIn32,
        MemDataIn16,MemDataIn8,
        MemDataOut128,MemDataOut64,MemDataOut32,
        MemDataOut16,MemDataOut8,
        LoadByteDisable,StoreByteDisable,
        shared,slot_shared,
        lookup,imap,
        TIEadd,TIEaddn,TIEcmp,TIEcsa,TIEmac,
        TIEmul,TIEmulpp,TIEmux,TIEpsel,TIEsel,
        TIEprint,
        PIFReadDataBits,PIFWriteDataBits,
        IsaMemoryOrder,IsaUseBooleans,IsaCoprocessorCount,
    },
    moredelim=[is][\bfseries]{\$\$\$}{\$\$\$},
}

\newcommand{\question}[1]{\noindent\textit{#1}\\}
\newcommand{\answer}[1]{\vspace{-1.5em}\begin{oframed}#1\end{oframed}}

\begin{document}

% ############################################################
% TODO: Fill in your names and e-mail addresses
\author{
Student One 
\emailaddress{student.one@mailbox.tu-dresden.de}
\and
Student Two
\emailaddress{student.two@mailbox.tu-dresden.de}
}
\date{30.09.2024}
% ############################################################

\faculty{Faculty of Electrical Engineering and Computer Engineering}
\institute{Institute of Communication Technology}
\chair{Vodafone Chair of Mobile Communications Systems}
\title{Project Report: HW/SW Co-Design Lab}
\subtitle{SS 2024}
\supervisor{Viktor Razilov \and Dr. Emil Mat\'{u}\v{s}}

\maketitle

\chapter*{Introduction}

\newlength{\unit}
\setlength{\unit}{.5cm}

\tikzset{
    start/.style={draw,rectangle,text width=2.5\unit,rounded corners=.2\unit,
    minimum height=1\unit,align=center},
    xtstep/.style={draw,rectangle,text width=4.5\unit,minimum height=\unit,
    align=center},
    fork/.style={draw,diamond,text width=2.9\unit,inner sep=1pt,align=center,
    aspect=2}
}

\begin{wrapfigure}{r}{0.3\textwidth}
    \centering
    \begin{tikzpicture}[node distance=.6\unit,font=\tiny]
        \node[start] (step1) {Start};
        \node[xtstep,below=of step1] (step2) {1. Profile Application to find Hot
        Spots};
        \node[xtstep,below=of step2] (step3) {2. Describe TIE Instruction Set
        Extensions};
        \node[xtstep,below=of step3] (step4) {3. Modify Application to use TIE};
        \node[fork,below=of step4] (step5) {TIE functionally correct?};
        \node[fork,below=of step5] (step6) {Target performance achieved?};
        \node[start,below=of step6] (step7) {Hardware Optimization};

        \node[coordinate,right=of step2] (loopback) {};

        \draw[->] (step1) -- (step2);
        \draw[->] (step2) -- node[coordinate,midway] (loop) {} (step3);
        \draw[->] (step3) -- (step4);
        \draw[->] (step4) -- (step5);
        \draw[->] (step5) -- node[right] {Yes} (step6);
        \draw[->] (step5) -| node[midway,right] {No} (loopback |- loop) -- (loop);
        \draw[->] (step6) -- node[right] {Yes} (step7);
        \draw[->] (step6) -| node[midway,right] {No} (loopback |- loop) -- (loop);
    \end{tikzpicture}
    \caption{Similar to Tensilica Instruction Extension (TIE) Language, User's
    Guide. 02/2014, p. 4.}
    \label{fig:flow}
\end{wrapfigure}

The HW/SW Co-Design Lab is intended to give a first practical access to this
topic and experiencing the potential of the conjoint design of hardware and
software. For this purpose, the reconfigurable processor flow from the company
\emph{Tensilica} is used (shown in \cref{fig:flow}). The tool suite provides
convenient means for analyzing and optimizing the performance of the software.
Especially the possibility to easily extend the processor's hardware
architecture by customer specific instructions using the \emph{Tensilica
Instruction Extension (TIE)} language is well suited to demonstrate the
interaction between hardware and software design.

Within the scope of this lab, the following 2 exercises should be done:

\begin{enumerate}
\item Implementation of a finitie impulse response (FIR) filter design including
    performance analysis; HW/SW optimization: e.g., by Fusion, SIMD extension,
    etc.; also consider different implementation alternatives for the FIR.
\item Improve the performance of a given fast Fourier transform (FFT)/inverse
    fast Fourier transform (IFFT) algorithm by using basic instruction extension
    concepts (Fusion/SIMD/FLIX/etc.). Also consider different implementations
    such as the Decimation in Time (DIT) and Decimation in Frequency (DIF)
    algorithm.
\end{enumerate}

\noindent\emph{The written report should only include the results of the second task and
need not exceed 3 pages (source code excerpts not included)!}\\

\noindent The detailed project description can be found at the lab webpage:\\
\url{https://bildungsportal.sachsen.de/opal/auth/RepositoryEntry/21521498113}

\chapter*{Source Code}

\section*{Source code C files (only excerpts from the parts that have been
changed; please highlight important parts).}

% ############################################################
% TODO: Replace with your source code
% \lstinputlisting may also be used to include your source code file directly
% Emphasize parts with $$$ ... $$$
\begin{lstlisting}
int main()
{
    // Test
    $$$printf("Hello World");$$$
}
\end{lstlisting}
% ############################################################

\section*{Source code TIE files}

% ############################################################
% TODO: Replace with your source code
% \lstinputlisting may also be used to include your source code file directly
\begin{lstlisting}[style=tie]
// Test
operation MUL_SRL_16 {out AR z, in AR a, in AR b} {
    wire [31:0] m = TIEmul(a[15:0],b[15:0],1);
    $$$assign z = {16'b0, m[31:16]};$$$
}
\end{lstlisting}

\chapter*{Questions}

% TODO: Answer question
\question{Which acceleration technique(s) has/have been used and why?}
\answer{
}

\question{Which part of the FFT algorithm did you accelerate and why?}
\answer{
}

\question{What is the impact on execution time (speedup)?}
\answer{
}

\question{What is the impact on the hardware (area)?}
\answer{
}

\question{What is the impact on the software (new instructions, modified source
code, compiler intrinsics)?}
\answer{
}

\chapter*{Detailed Report}

\question{Step-by-step description \textbf{(min 1. page, max 3 pages)} of each
optimization step that has been tried (even if no performance increase could be
achieved):
\begin{itemize}
    \item What software optimization or hardware acceleration strategies you
        tried?
    \item Refer your explanations to the C/TIE code making use of the line
        numbers
    \item How did the performance increase (or decrease)? If you obtain a
        performance decrease, explain why and due to which acceleration method.
    \item What are the costs for the performance increase?
    \item Make a tradeoff performance vs. increased costs.
\end{itemize}
Conclude with an overall summary: Which combination of optimization and hardware
acceleration strategies do you suggest, i.e., yields the best performance
results or speed/area tradeoff? Summarize the experiences you gained with this
task.}
\answer{
}

\end{document}
